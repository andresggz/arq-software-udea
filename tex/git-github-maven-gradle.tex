\documentclass[a4paper,11pt]{article}
\usepackage[colorlinks=true,linkcolor=blue,allcolors=red]{hyperref}

\renewcommand\thefootnote{\textcolor{red}{\arabic{footnote}}}

\title{Git \& GitHub, Maven \& Gradle}
\author{\\\\ Cristian Camilo Serna Betancur \\\\ Andres Grisales Gonzales}

\begin{document}
\maketitle
\tableofcontents

\section{Git}
Git is a \emph{free and open source} distributed version control system designed 
to handle everything from small to very large projects with speed and efficiency
\cite{git}. It was created by Linus Torvalds in 2005 for development of the 
\emph{Linux kernel}, with other kernel developers contributing to its initial 
development. Since 2005, Junio Hamano has been the core maintainer\cite{gita}.

It allows us to read the history of the project, its changes, and the thoughts 
of the person who worked in them \footnote{Although sometimes we don't write 
good commit messages, sometimes we don't use more than one line to communicate 
neither our thoughts nor the purpose of our changes.}. If something is wrong, 
we can use it to return at the time where all was working well. With Git we can 
also tag a specific state of our project with a version number if we think some 
work is done, reliable, and ready to use.

\section{Github}
GitHub, Inc. is an American multinational corporation that provides hosting for 
software development and version control using Git\cite{github}. It has many 
features such as pull request (PR), bug tracking, task manager, and wikis for 
every project. It is one of the most important hosting providers for free and 
open source projects, since it hosts very important ones.

With GitHub we can improve our own projects and help on different ones, giving 
an receiving feedback, report issues and share knowledge, and ideas. We can use
hosted continous integration services such as Travis CI with it that enables us
to test and ship our apps with confidence\cite{travis}.

Git $+$ GitHub offer us an easy way to work with many people at the same project
\footnote{Since I've learned how to use Git and GitHub (GitHub or GitLab or 
Bitbucket, who cares) I've not been able to figure a different way to work with 
more than one person at the same project.}, does'nt matter if even we are in 
different countries, we can share our code and ideas with everyone in the world.



\section{Maven}
Maven is a build automation tool used primarily for Java projects. Maven can also
be used to build and manage projects written in Java, C#, Ruby, Scala, and other 
languages\cite{Maven}. Maven is an excellent tool for Java developers and it is
also possible to use it to manage the lifecycle of your projects. As a life cycle 
management tool, Maven works around phases.

Maven handles all phases of the project lifecycle, including validation, code
generation, compilation, testing, packaging, integration testing, verification,
installation and deployment\footnote{Since the beginning of my university studies,
I have always worked with maven, that is the system with which I have felt most
comfortable and happy; Maven is simple.}.

In Maven a Project Object Model or POM is the fundamental unit of work in Maven.
It is an XML file that contains information about the project and configuration
details used by Maven to build the project. It contains default values for most
projects\cite{Apache}.

One of Maven's most useful resources is its dependency management support: 
you simply define the libraries on which your application depends and Maven locates
them (either in your local repository or in a central repository), downloads them, and
uses them to compile your code\footnote{You need to have an internet connection. 
Maven the first time you run start downloading things from the internet and if that can't,
it's useless.}.

\section{Gradle}
Gradle is a build automation tool for multi-language software development. It controls the
development process in the tasks of compilation and packaging to testing, deployment, and
publishing\cite{Gradle}.\footnote{Gradle build scripts are written using Groovy or Kotlin DSL
(Domain Specific Language). Gradle has great flexibility and allows us to make uses of other
languages and not just Java, it also has a very stable dependency management system.}.
Gradle is highly customizable and fast as it completes tasks quickly and accurately by reusing
outputs from previous runs, only processing inputs that have changes in parallel.

Some features of Gradle that we can highlight are the following: Collaborative debugging,
incremental construction, custom repository design, transitive dependencies, built-in Groovy
and Scala support, incremental compilation for Java, packaging and distribution of JAR,
WAR and EAR, daemon compiler.

Gradle allows you to build from microservices to mobile applications, it can be used by small
startups as well as large companies, as it helps teams develop, automate and deliver quality
software in a shorter time (depending on its complexity and prior planning of the development
process)\cite{Gradle}.

\section{Maven vs Gradle}
Both build systems offer built-in ability to resolve configurable repository dependencies and
can cache dependencies locally and download them in parallel. In the world of Java programming,
the leader in software development for a considerable time has been Maven. Today, the time has
come to leave behind the endless configuration XML files and make way for something more modern
and powerful like Gradle. Gradle is more powerful. However, there are times that you really do
not need most of the features and functionalities it offers. Maven might be best for small
projects, while Gradle is best for bigger projects\cite{DZone}.

\begin{thebibliography}{99}
    % https://git-scm.com/
    \bibitem{git} H.~Part1: \emph{Germa \TeX}, TUGboat Volume~9, Issue~1(1988) 
     % https://en.wikipedia.org/wiki/Git
    \bibitem{gita} H.~Part1: \emph{Germa \TeX}, TUGboat Volume~9, Issue~1(1988) 
    % https://en.wikipedia.org/wiki/GitHub
    \bibitem{github} H.~Part1: \emph{Germa \TeX}, TUGboat Volume~9, Issue~1(1988) 
    %https://github.com/marketplace/travis-ci
    \bibitem{travis} H.~Part1: \emph{Germa \TeX}, TUGboat Volume~9, Issue~1(1988) 
\end{thebibliography}

\end{document}

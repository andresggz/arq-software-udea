\documentclass[a4paper,11pt]{article}
\usepackage[colorlinks=true,linkcolor=blue,allcolors=red]{hyperref}

\renewcommand\thefootnote{\textcolor{red}{\arabic{footnote}}}

%\title{Git \& Github, Mave \& Gradle}
\author{Cristian Camilo Serna Betancur \\\\ Andres Grisales}

\begin{document}
%\maketitle

\section{Git}
Git is a \emph{free and open source} distributed version control system designed 
to handle everything from small to very large projects with speed and efficiency
\cite{git}. 

It allows us to read the history of the project, its changes, and the thoughts 
of the person who worked in them \footnote{Although sometimes we don't write 
good commit messages, sometimes we don't use more than one line to communicate 
neither our thoughts nor the purpose of our changes.}. If something is wrong, 
we can use it to return at the time where all was working well. With Git we can 
also tag a specific state of our project with a version number if we think some 
work is done, reliable, and ready to use.

\section{Github}
GitHub, Inc. is an American multinational corporation that provides hosting for 
software development and version control using Git\cite{github}. It provides 
many features such as pull request (PR), bug tracking, task manager, and wikis 
for every project. These things allow us to improve our own projects or helping 
different projects, giving an receiving feedback, report issues and share 
knowledge, and ideas.

Git $+$ GitHub offer us an easy way to work with many people at the same project
\footnote{Since I've learned how to use Git and GitHub (GitHub or GitLab or 
Bitbucket, who cares) I've not been able to figure a different way to work with 
more than one person at the same project.}, does'nt matter if even we are in 
different countries, we can share our code and ideas with everyone.

%GitHub allows us to work with many people from many different
%countries into the same project, sharing knowlegde, issues and ideas.

\begin{thebibliography}{99}
    % https://git-scm.com/
    \bibitem{git} H.~Part1: \emph{Germa \TeX}, TUGboat Volume~9, Issue~1(1988) 
    % https://en.wikipedia.org/wiki/GitHub
    \bibitem{github} H.~Part1: \emph{Germa \TeX}, TUGboat Volume~9, Issue~1(1988) 
\end{thebibliography}

\end{document}
